\documentclass[prb,twocolumn]{revtex4}
\usepackage{graphicx}
\usepackage{amsmath}
\usepackage{amssymb}
\usepackage{epstopdf}

\begin{document}
\title{Title}

\author{Name}
\affiliation {
Physics Department, Virginia Tech, Blacksburg, Virginia 24061, USA\\
}


\begin{abstract}
Absract: A few sentences summarizing the topics of each problem.  The document should be in APS PRB format.  
\end{abstract}

\maketitle

\section{Problem 1}

\noindent
Your answers should be written in prose with proper grammar and spelling to include:\\
1) A brief (few sentence) introduction discussing the problem to be addressed.\\
2) A brief explanation of the method(s) used.\\
3) Any equations you might find necessary in explanations (define all symbols).  Equations can be written inline, $E=mc^{2}$, or in array form:
\begin{eqnarray}
E=mc^{2}
\label{nameforequation}
\end{eqnarray}
4) At least one plot showing your results (Label all axes and explain the graph in the caption/text) \\
5) A brief analysis/discussion of results that shows you have answered (and understood) the problem.  \\

%To insert a figure, remove % from each line and put file fig1.eps in the same directory as the latex file
% Update for Overleaf version 2: The image file name should not have a suffix, e.g., fig1.eps should be fig

%\begin{figure}[h!]
%\centerline{\includegraphics [width=3 in] {fig1}} \caption{Plot of something interesting} \label{nameforfigure1}
%\end{figure}

\section{Problem 2}

\begin{eqnarray}
F=ma
\label{namefornewequation}
\end{eqnarray}

%To insert a figure, remove % from each line and put file fig2.eps in the same directory as the latex file

%\begin{figure}[h!]
%\centerline{\includegraphics [width=3 in] {fig2}} \caption{Plot of something really interesting} \label{nameforfigure2}
%\end{figure}

Make sure you include references to documents \cite{thecoursetext} you may have used.

\begin{thebibliography}{99}

\bibitem{thecoursetext} T. Pang, \emph{Introduction to Computational Physics}, Cambridge University Press (2006).

\end{thebibliography}

\end{document}
